\documentclass[a4paper]{article}
\usepackage{inputenc}
\usepackage[UTF8]{ctex}
% 控制标题页的布局及内容
\usepackage{titling}
% 页面控制
\usepackage{geometry}
\geometry{scale=0.80}

\title{\zihao{-0} \textbf{请~愿~书}}
\author{} \date{}
\renewcommand{\maketitlehookb}{} % 移除作者栏
\renewcommand{\maketitlehookc}{} % 移除日期栏

\begin{document}
\maketitle

\begin{flushleft}
	\textbf{尊敬的代老师:}
\end{flushleft}

春风将之,一年一度的“两会”也在北京顺利召开,而我想借着这春风阐述我一直以来
的一个“心病”。望代老师能够仔细阅读。下面是我在本校生活两年来的感悟以及经验教训,
因篇幅以及时间有限所以这封请愿书的内容还没有完全得到我的审阅,但所说所写皆为
肺腑之言,望代老师理解。

在去年我与同一寝室的成员发生了短暂的冲突,而这件事根本原因只是因为一些没有
说清楚的误会,到后来甚至爆发肢体冲突,对于这件事情间接滋生了我对于下面将要
阐述想法。在那天半夜我就在阳台听着他们隔壁传来的嬉笑声,我就一直在回忆以前的
事情,人为什么要活着、为了什么而生活?也许是我对当天的事情感到震惊所以一直
在想这些深奥的哲学问题,但这也无意中为我埋下了一颗种子。

在那天离校以后我一路都在想,回到家后也与父母进行了深度交流,我感觉在那时
我才真正地像一个16、7岁的高中生,我也想明白我在午夜未曾能解释的问题。
我可以这样说,我在那天所理解的人生问题远比我在学校所理解、感悟的多。
之前代老师也曾说过很多学生实习以后都跟换了个人一样,这又是为什么呢?
还不是因为受到现实社会的激化,在看过真正的生活后没有人不会担心自己的未来,
我想他们也多多少少明白了些道理。可我们呢?我们不能等到自己要去厂里上班了
才恍然醒悟,不能等到子弹打进自己的脑袋里才后悔,
人生只有一次从来都没有后悔药。

在第二天返校后当我回到教室他们趾高气昂的嘲笑,我更加坚定了这种想法,
对于过去的事情就暂且不谈,就像狼与羊一样他们天生就不能在一个屋檐下生活,
如果我只是为了改善与他们之间的关系而放弃自己原有的原则,
那我就没有必要来写这封“请愿书”了。虽然这些事情都是每个人所必须经历
的事情,可这些事情我也并不是不能调和,而最根本的原因是我所学习和生活的环境
不足以满足我对进步的渴求,都说“不想当将军的士兵,不是好士兵”那么我们也
同理,没有人不想让自己的生活比别人更差,自己的前途是自己争取来的从来
不是靠什么“歪门邪道”。

人是为了自己而学习的

我也曾想过与班里学习较好的同学一起组成一个学习小组,我也打探过其他人的看法,
但后来我因为参加技能大赛遂放弃了这个想法,所以这个想法也从来没有被代老师所知。
在技能大赛的这短时间我还是学到了很多知识,拓宽了我的视野,
我也了解到很多关于我们这个行业的相关信息,但我没有办法与同为计算机专业的
其他同学分享我所知道的事情,对此我表示遗憾但同时也感到愤慨,
他们对自己未来的专业竟毫不关心,平时的专业课上、偷摸玩游戏的、睡觉的这些
比比皆是相信代老师也清楚,更不用说最重要的文化课了,我对这样的环境感到失望,
都说要适应环境可我实在是不想在这种乌烟瘴气的教室里待下去了,我曾经试着
在教室认真跟着老师所讲的内容所学习,但这实在是太难了更何况我又参加了技能大赛
本身课程进度就落后于老师的进度,后来就决定从头开始学初中的课程也是在那个时候
我清晰地认识到自己的缺陷,我敢说在教室的同学们现在可能连初中的基础题
都不会写了,在那时我就开始“发疯”去补习我遗忘的知识,当我去联系我曾经的
同学时,我才发现
自己就是井底青蛙,除了会那点技术自己什么也不是,我也不甘心落后于曾经的朋友
在高一短暂的培训结束后的新学期,我仍想留在实训室利用手上的资源来学习,我有时
甚至连教室都不去了,但我承认我这样做是不对的,这就是旷课。
后来我“老实”了,因为
我的成绩并没有显著的提高,尽管我那时并没有开始复习高中所学的知识。但现在想想
我认为非常值得,方法虽然极端可也是迫不得已,我在班里根本就没有发言权,
而同学们
之间又互相勾心斗角,因为我去年在寝室发生的事情多少都在他们心中留下了
话柄,他们
可以在任何时候那这件事请来取笑我,我虽然对此表现出一种“无所谓”的态度,可心里
却一直耿耿于怀,毕竟没有人想听到对自己不想听的东西。但这些是根植在人心中的,
偏见也是如此,改变自己也不是一天两天就能完成的,归根结底改变自己是需要莫大
的勇气和坚强的毅力否则是不能“脱胎换骨”的。

对于我的努力我认为需要一个契机,怎样才能判断我是否用功呢?
我每天可以背十个单词、写书上的练习题、做20个俯卧撑,这些我都可以做到,
对于成绩我认为我能够有很大的提升,毕竟学没学到是自己的事情而考试则是检验自己
的测试,毕竟学习是自己的事情而不是为了突出自己在成绩上的“优越感”而下达的
虚伪的承诺。

我可以对着自己的良心发誓,他们敢吗?我宁愿去工厂磨练自己
我也不愿意跟他们同流合污,我的父母对我的抚养已经让他们的青丝长出了银发,
就算不是为了成绩我也要考虑我的父母。

我希望代老师再看完后能够理解我强烈的愿望以及我对当下的反思,
以上就是我的请求,希望得到代老师的理解。

\begin{flushright}
	\textbf{周逸轩} \\
    \textbf{二零二五年三月四日}
\end{flushright}

\end{document}
