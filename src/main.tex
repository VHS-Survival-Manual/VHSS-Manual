\documentclass{article}
% 基本中文环境
\usepackage[utf8]{inputenc}
\usepackage{ctex}
% 文本盒子
\usepackage{lipsum}
\usepackage[hooks]{tcolorbox}
% 链接
\usepackage{hyperref}
\hypersetup {
    colorlinks=true, linkcolor=cyan, filecolor=blue,
    urlcolor=red, citecolor=green
}

\begin{document}

\title{\textbf{中专生存手册}}
\author{\textbf{本书编委会}}
\date{\today{}}

\maketitle
\thispagestyle{empty}
\newpage

\section*{声明}

《中职生存手册》(简称“手册”)是一本公益手册。本手册为不同作者根据
自身经验自发撰写完成,版权属于本书编委会。在本书编写过程中,
我们始终以“中立”、“客观”以及“公益”为原则,
没有接受任何组织的任何形式的支持。未经编委会许可,任何住址或个人
不得违反相应的版权条例擅自更改、修改本书内容;不允许对本书原意进行曲解和大规模分发。

本手册作者不能保证手册内容中没有对其他组织的误解和偏见。手册内容的正确性均没有
经受权威审查,手册作者无法保证手册中的方法始终有效。手册作者亦
无法确认手册是否违反当地法律法规,
请各位读者参照当地行政规定。如有违反,请您立刻停止阅读该手册并销毁所有本手册
副本,对于未经授权传播手册而造成的各种问题,手册作者均不负责。
手册作者无法确认本手册内容是否会对读者身心产生影响。如果您因为阅读
该手册而产生不适,请您立刻停止阅读该手册并咨询心理医生。

在\textbf{Github}上,我们提供本书的最新电子版下载:
\href{https://github.com/VHS-Survival-Manual/VHSS-Manual}{VHSS-Manual}

\begin{flushright}
本书编委会
\end{flushright}

\newpage
\section*{序言}

在中专的三年时间里,我见过太多的是非、太多的荒诞、太多的茫然,而这一切只是
因为他们在中考没有考出理想的成绩或是因为国家政策而被迫“分流”,大好的青春年华却在
“职高”中昏昏沉沉地度过,每一天都在按部就班地生活,没有学习热情也没有技术情操,
每一天最期待的便是不上课、可以自由活动的时间,说白了就是不受学校束缚的时间,
晚上、玩一玩手机、刷刷视频看看毫无营养的东西然后再因为一些鸡毛蒜皮的小事而大打出手,
这不是学校这是个人力工厂,而我们每个人都是生活在《十八岁的流水线》中,
尽管我们还没有出镜我们还在工厂外面的教室里,可我们就是在“温水煮青蛙”的
环境下,而这环境就像一锅烂汤里面的主菜就是“老鼠屎”,便是当下中专与职高的情况。
我知道这太绝对,如果你也深陷其中那么你就会知道我说的是否属实,我必须承认,在
职高中肯定有“真正”学习的人,可那样的人又有多少?他们是真的在为自己学习还是
虚荣心作祟?这里便不做评价,我们也不去探讨他们的事情,因为这样只会使得我们
这种人的团体的加速分裂。

在这当中,来这里的原因很复杂,有靠关系的;有分太低但有钱的;有分数离普高仅差几分
却因为家庭负担而来的,总而言之,每个人在入校时都会隐藏一些秘密,而这一部分
秘密又成为他们心里挥之不去的阴影,而入学只是堕落的开始,毫不夸张地说入学就是
走进工厂的倒计时,我知道这里的描述有点极端,可谁能保证一个心中已决意改变
的人走入这样一个环境不会失去其本心呢?

我们要在这本手册中谈些什么?我们要你们如何在现实和理想之间找到完美的平衡点,
如何在现实找到前进的方向并为之奋斗,如何将梦想转变为现实,以及让你知道
怎样活出你想要的人生。我们知道这些东西并不好写,甚至可能写成“鸡汤文学”,
如果真写成这样那我们就违背“本心”了,我们希望这本手册能给身处黑暗中的
同志一些光亮指引他们的理想,在看完后希望他们再次拥有勇气、自信以及
面对现实的力量。
\pagenumbering{Roman}

\newpage
\tableofcontents
\pagenumbering{Roman}

\pagenumbering{arabic}
\section{励志篇}

\subsection{进入中职}

\begin{center}
    \emph{Welcome my son, welcome to the machine.\\————《Welcome to the machine》\textbf{Pink Floyd}}
\end{center}

\textbf{同志们!在这一部分我们要告诉大家并让大家知道什么是中专、
什么是职高以及什么是职教中心。}

首先,中专分为两种,一种是“普通中专”一种是“成人中专”,这里我们只探讨“普通中专”。
\textbf{普通中专}是实施全日制中等学历的职业学校,想必大家也清楚为什么来到这里而不是
普通高中,往后我们就要顶着这个标签一直到毕业,这种滋味可不好受。我们在这里不谈论中专
的历史,我们只需要知道自九十年代后期中专的地位便一落千尺,知道这些就够了。

那么“中等专业”和职业高中有什么区别呢?首先它们的生源都是来自初中应届毕业生,当然也包
括往届毕业生,这个就先另当别论。首先中专与职高并无具体差别,他们的学历都是一样的级别与
普通高中同级,我们是“中专学历”而他们的则是“普高学历”,两者都属于高级中学的范畴,

\subsection{熟悉“机器”}

\subsection{看看身边}

\subsection{还记得什么?}

\subsection{陌生的黑板}




\end{document}
