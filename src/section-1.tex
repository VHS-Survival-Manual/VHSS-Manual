\section{励志篇}

\subsection{进入中职}

\begin{center}
    \emph{Welcome my son, welcome to the machine.\\————《Welcome to the machine》\textbf{Pink Floyd}}
\end{center}

\textbf{同志们!在这一部分我们要告诉大家并让大家知道什么是中专、
什么是职高以及什么是职教中心。}

首先,中专分为两种,一种是“普通中专”一种是“成人中专”,这里我们只探讨“普通中专”。
\textbf{普通中专}是实施全日制中等学历的职业学校,想必大家也清楚为什么来到这里而不是
普通高中,往后我们就要顶着这个标签一直到毕业,这种滋味可不好受。我们在这里不谈论中专
的历史,我们只需要知道自九十年代后期中专的地位便一落千尺,知道这些就够了。

那么“中等专业”和职业高中有什么区别呢?首先它们的生源都是来自初中应届毕业生,当然也包
括往届毕业生,这个就先另当别论。首先中专与职高并无具体差别,他们的学历都是一样的级别与
普通高中同级,我们是“中专学历”而他们的则是“普高学历”,两者都属于高级中学的范畴,

\subsection{熟悉“机器”}

\subsection{看看身边}

\subsection{还记得什么?}

\subsection{陌生的黑板}


